\documentclass[12pt]{article}
\usepackage{graphicx}
\usepackage[spanish]{babel}
\usepackage{amsmath,amsthm,amssymb}

\title{Formulario}
\author{Sebastian PA}
\date{Julio 2023}

\begin{document}

\maketitle

\section*{Prestaciones}
\begin{equation}
    Prestaciones\hspace{2pt}X = \frac{1}{Tiempo\hspace{2pt}de\hspace{2pt}Ejecucion\hspace{2pt}X}
\end{equation}  

\begin{equation}
    Prestaciones\hspace{2pt}X > Prestaciones\hspace{2pt}Y
\end{equation}

\begin{equation}
    \frac{1}{Tiempo\hspace{2pt}de\hspace{2pt}Ejecucion\hspace{2pt}de\hspace{2pt}X} > \frac{1}{Tiempo\hspace{2pt}de\hspace{2pt}Ejecucion\hspace{2pt}de\hspace{2pt}Y}
\end{equation}

\begin{equation}
    Tiempo\hspace{2pt}de\hspace{2pt}Ejecucion\hspace{2pt}de\hspace{2pt}X > Tiempo\hspace{2pt}de\hspace{2pt}Ejecucion\hspace{2pt}de\hspace{2pt}Y
\end{equation}

\begin{equation}
    \frac{Prestaciones\hspace{2pt}X}{Prestaciones\hspace{2pt}Y} = n
\end{equation}

\begin{equation}
    \frac{Prestaciones\hspace{2pt}X}{Prestaciones\hspace{2pt}Y} = \frac{Tiempo\hspace{2pt}de\hspace{2pt}Ejecucion\hspace{2pt}de\hspace{2pt}Y}{Tiempo\hspace{2pt}de\hspace{2pt}Ejecucion\hspace{2pt}de\hspace{2pt}X} = n
\end{equation}

\subsection*{Prestaciones de la CPU y sus factores}

\begin{equation}  Tiempo\hspace{2pt}de\hspace{2pt}ejec.\hspace{2pt}CPU=Ciclos\hspace{2pt}de\hspace{2pt}reloj\hspace{2pt}de\hspace{2pt}CPU\hspace{2pt}\times\hspace{2pt}Tiempo\hspace{2pt}del\hspace{2pt}Ciclo\hspace{2pt}del\hspace{2pt}Reloj
\end{equation}

\begin{equation}
    Tiempo\hspace{2pt}de\hspace{2pt}ejecucion\hspace{2pt}de\hspace{2pt} CPU = \frac{Ciclos\hspace{2pt}de\hspace{2pt}reloj\hspace{2pt}de\hspace{2pt}CPU}{Frecuencia\hspace{2pt}del\hspace{2pt}reloj}
\end{equation}

\subsection*{Prestaciones de las Instrucciones}

\begin{equation}
    Ciclos\hspace{2pt}de\hspace{2pt}reloj\hspace{2pt}de\hspace{2pt}CPU = Instrucciones\hspace{2pt}de\hspace{2pt}un\hspace{2pt}programa\hspace{2pt}\times\hspace{2pt}Media\hspace{2pt}de\hspace{2pt}Ciclos\hspace{2pt}por\hspace{2pt}Instruccion
\end{equation}

\subsection*{La ecuación clásica de las prestaciones de la CPU}

\begin{equation}
    Tiempo\hspace{2pt}de\hspace{2pt}ejecucion = Numero\hspace{2pt}de\hspace{2pt}instrucciones\hspace{2pt}\times\hspace{2pt}CPI\hspace{2pt}\times\hspace{2pt}Tiempo\hspace{2pt}de\hspace{2pt}Ciclo
\end{equation}

\begin{equation}
    Ciclos\hspace{2pt}de\hspace{2pt}reloj\hspace{2pt}CPU = \sum_{i=1}^{n}(CPI_i\hspace{2pt}\times\hspace{2pt}C_i)
\end{equation}

\begin{equation}
    CPI = \frac{Ciclos\hspace{2pt}de\hspace{2pt}reloj\hspace{2pt}de\hspace{2pt}CPU}{Numero\hspace{2pt}de\hspace{2pt}instrucciones}
\end{equation}

\begin{equation}
    Tiempo = \frac{Segundos}{Programa} = \frac{Instrucciones}{Programa} \times \frac{Ciclos\hspace{2pt}de\hspace{2pt}Reloj}{Instruccion} \times \frac{Segundos}{Ciclos\hspace{2pt}de\hspace{2pt}Reloj}
\end{equation}

\section*{El muro de la potencia}

\begin{equation}
    Potencia = Carga\hspace{2pt}Capacitiva \times Voltaje_2 \times Frecuencia\hspace{2pt}de\hspace{2pt}conmutacion
\end{equation}

\subsection*{Casos reales: fabricación y evaluación del AMD Opteron x4 }
\subsection*{Coste de un Circuito Integrado}

\begin{equation}
    Coste\hspace{2pt}por\hspace{2pt}dado = \frac{Coste\hspace{2pt}por\hspace{2pt}oblea}{Dado\hspace{2pt}por\hspace{2pt}oblea \times Factor\hspace{2pt}de\hspace{2pt}produccion}
\end{equation}

\begin{equation}
    Dados\hspace{2pt}por\hspace{2pt}Oblea = \frac{Area\hspace{2pt}de\hspace{2pt}la\hspace{2pt}oblea}{Area\hspace{2pt}del\hspace{2pt}dado}
\end{equation}


\begin{equation}
    Factor\hspace{2pt}de\hspace{2pt}Produccion = \frac{1}{(1 + (Defectos\hspace{2pt}por\hspace{2pt}area \times Area\hspace{2pt}del\hspace{2pt}dado / 2))^2}
\end{equation}

\subsection*{Evaluación de la CPU con programas de prueba SPEC}    

\begin{equation}
    n \sqrt{\prod_{i=1}^{n} Relaciones\hspace{2pt}de\hspace{2pt}tiempos\hspace{2pt}de\hspace{2pt}ejecucion_i}
\end{equation}

\begin{equation}
    ssjops\hspace{2pt}global\hspace{2pt}por\hspace{2pt}vatio = (\sum_{i = 0}^{10} ssjops_i )/(\sum_{i=0}^{10} potencia_i )
\end{equation}

\section*{Falacias y errores habituales}
\subsection*{Ley de Amdahl}

\begin{equation}
    Tiempos\hspace{2pt}ejec.\hspace{2pt}despues\hspace{2pt}mejoras = \frac{Tiempo\hspace{2pt}ejec.\hspace{2pt}por\hspace{2pt}mejora}{Cantidad\hspace{2pt}mejora} + Tiempo\hspace{2pt}ejec.\hspace{2pt}no\hspace{2pt}afectado
\end{equation}

\subsection*{MIPS}

\begin{equation}
    MIPS = \frac{Número\hspace{2pt}de\hspace{2pt}instrcciones}{Tiempo\hspace{2pt}de\hspace{2pt}ejecución\times10^6}
\end{equation}

\begin{equation}
    MIPS = \frac{Número\hspace{2pt}de\hspace{2pt}instrucciones}{\frac{Número\hspace{2pt}de\hspace{2pt}instrucciones\times CPI}{Frecuencia\hspace{2pt}de\hspace{2pt}reloj}\times 10^6} = \frac{Frecuencia\hspace{2pt}de\hspace{2pt}reloj}{CPI \times 10^6}
\end{equation}
\end{document}